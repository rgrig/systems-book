\usepackage{amsthm}
\usepackage{booktabs}
\usepackage[T1]{fontenc}
\usepackage[margin=1in]{geometry}
\usepackage{mathtools}
\usepackage{microtype}
\usepackage{minted}
\usepackage{natbib}
\usepackage{newpxmath}
\usepackage{newpxtext}
\usepackage{rgalg}
\usepackage{thmtools}
\usepackage{tikz}
\usepackage{xcolor}

\usepackage{hyperref}

\usepgflibrary{arrows}
\usetikzlibrary{matrix}
\usetikzlibrary{positioning}

\definecolor{darkblue}{rgb}{0,0,0.4}
\definecolor{darkgreen}{rgb}{0,0.4,0}
\hypersetup{colorlinks,linkcolor=darkblue,citecolor=darkblue,urlcolor=darkblue}

\usemintedstyle{tango}
\newminted{c}{linenos}
\newminted{java}{linenos}

\renewcommand*\.[1]{{\sf#1}}

\newcommand*\lecture[1]{
  \centerline{
  \framebox[1.05\textwidth]{\vbox{
    \hbox to\textwidth{{\bf CO527: Operating Systems}\hfil Spring 2017}
    \vspace{3ex}
    \hbox to\textwidth{\hfil\Large\bf#1\hfil}
    \vspace{3ex}
    \hbox to\textwidth{\it Radu Grigore\hfil}
  }}}
  \vspace{5ex}\par
}

\newcommand*\asgn\coloneqq
\newcommand*\df\emph
\newcommand*\todo[1]{\marginpar{\raggedright\tiny\color{darkgreen}TODO: #1\par}}
\newcommand*\defeq\coloneqq

\declaretheorem{theorem}
\declaretheorem[sibling=theorem]{lemma}

\title{Operating Systems Concepts}
\author{Radu Grigore}

\begin{document}
\lecture{Implementing File Systems}

\section*{Virtual File System}

In the previous lecture,
  we saw that UNIX-like operating systems pretend there is only one tree of files.
We can identify nodes of this tree by paths, which are strings.
Regular files are leafs of this tree.
Once we get a handle on a regular file,
  we can open it, read it, modify it, and close it.
More interestingly, to the tree of files,
  we can add whole subtrees, by mounting real file systems.
Let us peek under the hood to see how Linux implements these operations.

The following three data structures correspond to several concepts we saw:
\begin{center}
\begin{tabular}{@{}lll@{}}
\toprule
Concept & Data Structure & Defined in \\
\midrule
path step & \.{struct dentry} & \.{include/linux/dcache.h} \\
tree node & \.{struct inode} & \.{include/linux/fs.h} \\
open file & \.{struct file} & \.{include/linux/fs.h} \\
\bottomrule
\end{tabular}
\end{center}
These data structures are created on-demand, from the data existing on disk:
If one wants to open a file \.{/a/b/c},
  then several dentries, several inodes, and a file are created by the kernel.
Once created, these data structures are kept around in memory for a while,
  in a \emph{dcache} ({\bf d}irectory {\bf cache})
    which maps path names to corresponding dentries.

Recall that for a path like \.{/a/b/c}
  we said that \.{a}, \.{b} and \.{c} are path steps.
In our tree picture,
  path steps correspond to labelled edges.
Each \.{dentry} ({\bf d}irectory {\bf entry}) contains
  its label,
  information about its neighbours in the tree,
  and a pointer to the tree node it points to.
\begin{ccode}
struct dentry {
  struct qstr d_name;  // label on the edge; e.g. "a" in the  /a/b/c example
  struct dentry * d_parent; // edge from above
  struct list_head d_child; // sibling edges
  struct list_head d_subdirs; // edges from below
  struct inode * d_inode; // tree node, at the low end of the edge
  // ...
};
\end{ccode}
Each \.{inode} contains information about a file,
  such as its owner, permissions, and modification times.
\begin{ccode}
struct inode { kuid_t i_uid; kgid_t i_gid; /* ... */ };
\end{ccode}
A user space file handle is essentially a reference to a kernel \.{struct file}.
Each \.{file} tracks the current offset:
\begin{ccode}
struct file { struct inode * f_inode;  loff_t f_pos; /* ... */ };
\end{ccode}

These data structures give a uniform way to represent the tree of files;
  that is, the virtual file system.
But, recall that this tree is a patchwork of subtrees that correspond
  to real file systems.
How are these different file systems mixed?
Each of the data structures from above remembers how to perform operations
  specific to the real file system they are a part of:
\begin{ccode}
struct dentry { const struct dentry_operations * d_op; /* ... */ };
struct inode { const struct inode_operations * i_op; /* ... */ };
struct file { const struct file_operations * f_op; /* ... */ };
\end{ccode}
For example, the file operations include
\begin{ccode}
struct file_operations {
  int (*open)(struct inode *, struct file *);
  ssize_t (*read)(struct file *, char *, size_t, loff_t *);
  int (*iterate)(struct file *, struct dir_context *);
  // ...
};
\end{ccode}
One could view this scheme as a way to do inheritance in C\null.

If one creates a file, for example, the inode operation \.{create} gets called.
This operation will be specific to the real file system
  within which we create a file.
The operation creates an \.{inode}, and populates its \.{i\_op} field.
If the new \.{inode} is a directory,
  then its \.{create} operation was likely set to be the same
  as the \.{create} operation of the parent.
But, in the virtual file system we sometimes cross boundaries
  from one real file system to another real file system.
This happens at \df{mount points}.
So, let us talk briefly about mounting.

Each file system, such as ext3 or VFAT or MINIX,
  is described by a \.{struct file\_system\_type}.
\begin{ccode}
struct file_system_type {
  const char * name;
  struct dentry * (*mount)(struct file_system_type *, int, const char *, void *);
  // ...
};
\end{ccode}
You can see the names of the registered file system types by looking
  at the file \.{/proc/filesystem}.
(Incidentally, \.{/proc} is the mount point of a file system of type \.{procfs},
  where the kernel publishes all kinds of useful information.)
The main characteristic of a file system type, however, is not its name,
  but its \.{mount} function.
The \.{mount} function is a way to obtain the \.{dentry}
  that is the root of a real file system,
  also known as its mount point.
The third argument could be, for example, the string `\.{/dev/sda1}'.
Apart from returning the root \.{dentry},
  the \.{mount} function also creates an instance of \.{struct super\_block}.
\begin{ccode}
struct super_block { const super_operations * s_op; /* ... */ };
struct super_operations {
  struct inode * (*alloc_inode)(struct super_block * sb);
  void (*delete_inode)(struct inode *);
  void (*dirty_inode)(struct inode *, int);
  int (*write_inode)(struct inode *, int);
  // ...
};
\end{ccode}
There is one instance of \.{struct super\_block} for each mounted file system.
Data is actually written to the backing store by the super operation
  \.{write\_inode}.


\section*{MINIX File System}

Now let us look in some detail at how one particular file system is implemented.
MINIX is not widely used ---
  we look at it because its implementation is simple,
  in comparison with the other file system types.
Most of the relevant files are in \.{fs/minix/}.
Several data structures related to the physical layout are in
  \.{include/uapi/linux/minix\_fs.h}.

The \.{struct file\_system\_type} for MINIX is declared in \.{fs/minix/inode.c}.
\begin{ccode}
static struct file_system_type minix_fs_type {
  .name = "minix",
  .mount = minix_mount,
  .fs_flags = FS_REQUIRES_DEV,
  // ...
};
\end{ccode}
From here we find out that MINIX file systems require a device.
A device is typically identified by an absolute path in the virtual file system.
It could be something like the block special file \.{/dev/sda1},
  or it could be a regular file.
The function \.{minix\_mount} is defined in the same file.
It delegates most of the work to the generic \.{mount\_bdev},
  where \.{bdev} stands for block device.
The function \.{mount\_bdev} is given a callback,
  the function \.{minix\_fill\_super},
  which does the MINIX-specific work.
Let us take a look at it.

The first thing it does is to read block~$1$ (not~$0$, which is for booting):
\begin{ccode}
static int minix_fill_super(struct super_block * s, void * data, int silent) {
  // ...
  if (!(bh = sb_bread(s, 1))) goto bad_sb;
  ms = (struct minix_super_block *) bh->b_data;
\end{ccode}
where \.{minix\_super\_block} is defined in \.{minix\_fs.h}:
\begin{ccode}
struct minix_super_block {
  __u16 s_ninodes;
  __u16 s_nzones;
  // ...
};
\end{ccode}
There are two things to note here.
First, we were able to call the function \.{sb\_read} to get a buffer for block~$1$:
  this was possible because of the work done by \.{mount\_bdev}.
Most of the interaction with the backing store is done through these \.{sb\_*}
  functions, which read/write whole blocks, and cache them in memory, for speed.
A block is often (but not always) the same size as a page: $4$~KiB.
Second, \.{struct minix\_super\_node} defines the physical layout of the superblock.
Indeed, the next thing that happens is that the information is copied
  from \.{struct minix\_super\_node} into the in-memory representation
  of a MINIX super node.
\begin{ccode}
  sbi = kzalloc(sizeof(struct minix_sb_info), GFP_KERNEL);
  s->s_fs_info = sbi;
  // ...
  sbi->s_ninodes = ms->s_ninodes;
  sbi->s_nzones = ms->s_nzones;
  sbi->s_imap_zones = ms->s_imap_zones;
  sbi->s_zmap_zones = ms->s_zmap_zones;
  // ...
\end{ccode}
The next thing being done is that $\.{s\_imap\_zones}+\.{s\_zmap\_zones}$ blocks
  are loaded in memory.
\begin{ccode}
  block = 2;
  for (i = 0; i < sbi->s_imap_zones; i++, block++) sbi->s_imap[i] = sb_bread(s, block);
  for (i = 0; i < sbi->s_zmap_zones; i++, block++) sbi->s_zmap[i] = sb_bread(s, block);
\end{ccode}
Finally, the root inode is read.
\begin{ccode}
  root_inode = minix_iget(s, MINIX_ROOT_NODE); // MINIX_ROOT_NODE is 1
  s->s_root = d_make_root(root_inode);
\end{ccode}
The function \.{d\_make\_root} returns a \.{dentry},
  which is the one eventually returned by \.{minix\_mount}.
The function \.{minix\_iget} first invokes \.{iget\_locked},
  which searches a global cache for inodes,
  and then if that fails it reads from the device using \.{V2\_minix\_iget},
  which in turn invokes \.{minix\_V2\_raw\_inode}.
This latter function does some interesting offset computation:
\begin{ccode}
struct minix2_inode *
minix_V2_raw_inode(struct super_block * sb, ino_t ino, struct buffer_head ** bh) {
  struct minix_sb_info * sbi = minix_sb(sb);
  int minix2_inodes_per_block = sb->s_blocksize / sizeof(struct minix2_inode);
  // ...
  ino--;
  int block = 2 + sbi->s_imap_blocks + sbi->s_zmap->blocks
    + ino / minix2_inodes_per_block;
  *bh = sb_bread(sb, block);
  // ...
  struct minix2_inode * p = (void*) (*bh)->b_data;
  return p + ino % minix2_inodes_per_block;
};
\end{ccode}
OK, maybe not that interesting, but at least not completely trivial.
You should definitely make sure you understand what happens above.

\smallskip

It turns out that the two groups of blocks at the beginning are bitmaps:
  the first \.{s\_imap\_blocks} track which inodes are used,
  the next \.{s\_zmap\_blocks} track which data blocks are used.

\section*{Tracking Free Space}

After so much code, let us step back into idea-land.
File systems, like memory allocators,
  need to track which space is available and which space is used.
What is specific to file systems is that allocation is done in big chunks.
Yes, even a one-byte file occupies a whole block, which takes a few kilobytes.
Because allocation is done in big chunks,
  file systems can use simpler methods of tracking free space.
\todo{more explanations}

Consider a file system with $2^n$~data blocks, each holding $2^k$~bits.
We can track if one of these data blocks is used with $1$~bit.
Thus, we need $2^n$~bits, which fit in $2^{n-k}$ extra blocks.
For example, if there are $8192$~data blocks, each of $1024$~\emph{bytes},
  then we can track which blocks are used by using \emph{one} extra block.
This scheme is used by MINIX\null.

Alternatively, one can simply list the blocks that are free, in a list.
Consider again a file system with $2^n$~data blocks, each holding $2^k$~bits;
  and \emph{no} extra blocks this time.
To identify a block, we need a number from $\{0,1,2,\ldots,2^n-1\}$,
  which we can store in $n$~bits.
Thus, in one block we can fit $\lfloor 2^k/n\rfloor$ block addresses.
If there are fewer free blocks, then we can store all their addresses in one block;
  say, in block~$0$.
If there are $\sim 2^n$ free blocks,
  then we need $\sim 2^n/(2^k/n)=2^{n-k}/n$ blocks to keep track of free blocks.
We can chain these blocks in a list.
Thus, we store only $\lfloor 2^k/n \rfloor - 1$ free block addresses in one block:
  the extra address is of the next block in the free list.
How does the free list end?
For example, by saying that the next block in it is block~$0$,
  which we have designated as the head of the list.


\section*{The Google File System}

Last time, we also saw file systems that are distributed among many computers,
  such as GFS (Google File System) and HDFS (Hadoop File System).
Now we look at how GFS works.

As seen in \autoref{fig:gfs},
  GFS is implemented by several Linux machines,
  one master and many chunk-servers.
A chunk is a 64~MiB portion of a file.
All files stored in GFS are split into such chunks,
  which are stored as normal Linux files on chunk-servers.
The master handles all operations that require coordination.
For example,
  all chunks have 64~bit handles which are produced by the master only,
  to ensure that they are unique.
When a client machine wants to perform an operation on a file
  (read, write, record append, \dots),
  it first asks the master for the chunk handle of the portion of interest.
Together with the chunk handle,
  the client also receives a list of chunk-server addresses
  where the chunk is stored.
From now on, the client interacts directly with chunk-servers.

\begin{figure} % fig:gfs <<<
\begin{center}
\includegraphics[width=0.8\textwidth]{gfs-arch.jpg}
\end{center}
\caption{GFS Architecture
\textcolor{darkgreen}{[TODO: redraw]}
}\label{fig:gfs}
\end{figure}
% >>>

GFS aims to work on cheap hardware, which fails relatively often.
How is it then justified to have a single point of failure --- the master?
First,
  note that the master needs not be cheap,
  because it is one computer out of hundreds or thousands.
Second,
  having a single master greatly simplifies the design of the whole system.
For example, let us consider concurrency issues.
Because hardware fails,
  each chunk is stored in multiple replicas.
Because there are multiple replicas,
  possibly on different continents,
  users may concurrently modify replicas of the same chunk.
Because the replicas are in different locations,
  it is difficult to say which modification occurred first.
Solutions that rely on looking at the clock have an obvious problem:
  which clock?
(And, if we are to get technical, it is in fact \emph{impossible}
  to say that one operation occurred before the other, because of special relativity.
According to it, if the events happen in different places,
  then we can come up with two observers, which move around really fast,
  for which the two events are ordered differently.)
Because of this difficulty in finding which write happened first,
  GFS makes no guarantees.
If it is somehow obvious that one write was first
  (say, there's a day distance in-between),
  then you will get what you expect.
If GFS thinks it's not obvious, then it won't try hard:
  it just picks \emph{some} order.
Now think about a write operation that spans several chunks.
GFS may even choose for each chunk a different order.
Thus, you end up with a jumble of bytes from the two different write operations.
This state of affairs is somehow considered tolerable,
  because users can use record append operations if they want better guarantees,
  as we saw last time.
So, confusion about which byte should be at some position in a file is tolerable.
But, confusion about which files \emph{exist} is a whole new level of wtf.
And that's exactly what we would get
  if the operations of file creation and file deletion were distributed.
The master solves this problem.
We don't want confusion about which files exist,
  so we force create and delete operations to go through the master.
The clock of the master is the arbiter.
In fact,
  chunk creation also has to go through the master,
  and chunk handles are essentially the master's clock when the chunk was created.

OK, but is well known that putting stuff on one computer
  is easier than distributing it.
This, however, does nothing to assuage our worries that the master
  is a single point of failure.
So, here's the third reason:
  We can easily take snapshots of the master's state,
  and copy them on different computers.
If the master fails, we can recover its state from one of these backups.

What is the master's state?
One thing to realise is that the master is there to perform a series of operations
  one after another, in order.
(The order part is important, as we saw when we discussed file creation and deletion.)
So, we could create a log which is a list of all the operations performed.
We can then replay this log, to recreate the master's state.
This is the basic approach taken by GFS, with some optimisations.

If the log becomes too long, it might be impractical to replay it.
Instead,
  one can take a snapshot of the data-structures that the master has in memory.
This is not trivial.
Dumping to disk the whole data-structure takes time,
  and you can't just stop the whole distributed file system while you do it.
So, one has to use persistent data-structures,
  as the ones common in functional languages.
Most of the state is kept in dictionaries, which map keys to values
For example,
  one dictionary maps file paths to lists of chunk handles;
  another dictionary maps chunk handles to lists of chunk-server addresses.
An efficient data structure for such dictionaries that can be made persistent,
  is the B-tree,
  and GFS uses a modification thereof.
\todo{Mention GC.}
\todo{How data is moved, in a pipe.}


\section*{Exercises}

\begin{enumerate}
\item
  $\blacktriangleright$
  Most of the information in \.{inode}s is readily available via the command \.{stat}.
  Read `\.{man stat}'.
\item
  $\blacktriangleright$
  What is an \.{inode}?
  What is a \.{dentry}?
  What is a \.{super\_block}?
\item
  Since a file can be memory mapped via \.{mmap},
    it must be that the virtual file system interacts with the virtual memory.
  How does this interaction work?
\item
  Read `\.{man mount}'.
\item
  $\blacktriangleright$
  Consider a file system with $10^6$~blocks, each of $1024$~bytes:
  $k$~blocks are reserved for a bitmap that tracks which data blocks are in use;
  the other $10^6-k$~blocks are data blocks.
  What is the minimum value of~$k$?
\item
  $\blacktriangleright$
  Consider a file system with total size $16$~GiB, and blocks of $1$~KiB.
  It tracks free space using the free list method.
  How many block addresses fit in one block?
\item
  Which method for tracking free space in a file system would you use?
  Why?
  [Hint (not answer!): What happens to the free list when the file system fills up?]
\item
  Suppose a GFS cluster is made out of 999~chunk-servers and 1~master.
  Initially, it was planned to spend \pounds100 on each computer.
  An engineer, however, requested that \pounds3000 are spent on the master,
    while still paying only \pounds100 for each chunk-server.
  By how much did the budget increase, as a percent?
\end{enumerate}

\section*{References}

\begin{itemize}
\item[{[1]}]
  Linux kernel 4.2.0,
  \.{Documentation/filesystems/vfs.txt},
  \.{fs/*},
  and \.{fs/minix/*}
\item[{[2]}]
  Ghemawat et al., \emph{The Google File System}, 2003
\end{itemize}


\end{document}

% vim:spell:spelllang=en_gb:
